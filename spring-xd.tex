% THIS IS SIGPROC-SP.TEX - VERSION 3.1
% WORKS WITH V3.2SP OF ACM_PROC_ARTICLE-SP.CLS
% APRIL 2009
%
% It is an example file showing how to use the 'acm_proc_article-sp.cls' V3.2SP
% LaTeX2e document class file for Conference Proceedings submissions.
% ----------------------------------------------------------------------------------------------------------------
% This .tex file (and associated .cls V3.2SP) *DOES NOT* produce:
%       1) The Permission Statement
%       2) The Conference (location) Info information
%       3) The Copyright Line with ACM data
%       4) Page numbering
% ---------------------------------------------------------------------------------------------------------------
% It is an example which *does* use the .bib file (from which the .bbl file
% is produced).
% REMEMBER HOWEVER: After having produced the .bbl file,
% and prior to final submission,
% you need to 'insert'  your .bbl file into your source .tex file so as to provide
% ONE 'self-contained' source file.
%
% Questions regarding SIGS should be sent to
% Adrienne Griscti ---> griscti@acm.org
%
% Questions/suggestions regarding the guidelines, .tex and .cls files, etc. to
% Gerald Murray ---> murray@hq.acm.org
%
% For tracking purposes - this is V3.1SP - APRIL 2009

\PassOptionsToPackage{pdfpagelabels=false}{hyperref}
\setlength{\paperheight}{11in}
\setlength{\paperwidth}{8.5in}
\documentclass{acm_proc_article-sp}
\usepackage{hyperref}
\usepackage{url}
\usepackage{mdwlist}

% This gives LaTeX permission to insert extra line breaks instead of
% trying to fit too much in a single line. This helps with the
% "Overfull \hbox" warnings (but so far doesn't eliminate all of them)
\pretolerance=3500

% This inserts an ugly black box next to blatant violations of the 
% overfull hbox rule
\overfullrule=2cm

% Remove the blank space for copyright; see http://www.acm.org/sigs/publications/sigfaq#a21
\makeatletter
\let\@copyrightspace\relax
\makeatother

\begin{document}

\title{Spring XD: A Modular Distributed Stream and Batch Processing System}

% You need the command \numberofauthors to handle the 'placement
% and alignment' of the authors beneath the title.
%
% For aesthetic reasons, we recommend 'three authors at a time'
% i.e. three 'name/affiliation blocks' be placed beneath the title.
%
% NOTE: You are NOT restricted in how many 'rows' of
% "name/affiliations" may appear. We just ask that you restrict
% the number of 'columns' to three.
%
% Because of the available 'opening page real-estate'
% we ask you to refrain from putting more than six authors
% (two rows with three columns) beneath the article title.
% More than six makes the first-page appear very cluttered indeed.
%
% Use the \alignauthor commands to handle the names
% and affiliations for an 'aesthetic maximum' of six authors.
% Add names, affiliations, addresses for
% the seventh etc. author(s) as the argument for the
% \additionalauthors command.
% These 'additional authors' will be output/set for you
% without further effort on your part as the last section in
% the body of your article BEFORE References or any Appendices.

\numberofauthors{5} %  in this sample file, there are a *total*
% of EIGHT authors. SIX appear on the 'first-page' (for formatting
% reasons) and the remaining two appear in the \additionalauthors section.
%
\author{
% You can go ahead and credit any number of authors here,
% e.g. one 'row of three' or two rows (consisting of one row of three
% and a second row of one, two or three).
%
% The command \alignauthor (no curly braces needed) should
% precede each author name, affiliation/snail-mail address and
% e-mail address. Additionally, tag each line of
% affiliation/address with \affaddr, and tag the
% e-mail address with \email.
%
% 1st. author
\alignauthor Sabby Anandan
% 2nd. author
\alignauthor Marius Bogoevici
% 3rd. author
\alignauthor Glenn Renfro
\and  % use '\and' if you need 'another row' of author names
% 4th. author
\alignauthor Ilayaperumal Gopinathan
% 5th. author
\alignauthor Patrick Peralta
}
% There's nothing stopping you putting the seventh, eighth, etc.
% author on the opening page (as the 'third row') but we ask,
% for aesthetic reasons that you place these 'additional authors'
% in the \additional authors block, viz.

% Just remember to make sure that the TOTAL number of authors
% is the number that will appear on the first page PLUS the
% number that will appear in the \additionalauthors section.

\maketitle
\begin{abstract}
Spring XD is a unified, distributed, and extensible system for data ingestion,
real time analytics, batch processing, and data export. The objective of
Spring XD is to simplify the development and deployment of streaming and batching
data applications. Spring XD is an Apache 2 licensed open source project developed
by Pivotal.  This paper discusses the motivation, architecture, and use cases
for Spring XD.
\end{abstract}

\input{spring-xd-introduction.tex}

\input{spring-xd-internal-architecture.tex}

\section{Modules}
\label{sec:Modules}
A Spring XD Module \cite{modules} is a data processing unit and as mentioned above
there are four types: source, processor, sink, and job. Modules in Spring XD are defined 
in their own application context. This allows for easy encapsulation and life cycle 
management for modules. Additionally, the use of an application context allows for easy 
module expansion.  Spring XD uses Spring Integration \cite{spring-integration-reference} 
as its foundation for implementing modules. A module is comprised of components that
implement the data processing logic and one or more connectors (known as channels)
that connect to the underlying message bus.

\par

Spring XD offers a suite of 23 sources, 24 sinks, 9 processors and 9 jobs that are ready 
to use at startup.  These modules integrate with a variety of well known and popular
data stores and processing systems such as JDBC, HDFS, MongoDB, Spark, Kafka, RabbitMQ,
Sqoop etc.,  If an existing module does not meet the needs of a given use case, Spring XD
supports custom modules.

Spring XD sink and source modules are Message Endpoints 
\cite{enterprise-integration-pattern-message-endpoint} 
that are responsible for sending data to and receiving data from external applications
respectively. A source is the entry point for data into the stream. A sink is
the module that dispatches the stream's results to an external application or storage system.
A processor module is used to modify data transmitted from the source to sink.
Multiple processors may be chained together. Batch jobs are used to execute batch
processing steps on a set of data.

\par

\subsection{Source}
Source modules receive inbound data and send to downstream modules in the stream or to a batch job
which could be triggered with the data. There are two source types: poller and event driven.
A poller source is based on the polling consumer pattern \cite{enterprise-integration-pattern-pollingconsumer}.
It polls an external application (such as a web service, FTP server, database) for data at a
configurable interval. An event driven source is based on the event driven
consumer pattern \cite{enterprise-integration-pattern-eventdrivenconsumer} which
opens a port to listen for incoming data that is pushed from an external application.

\par

In the case of a source module there is an ``output'' connector channel to dispatch data
transmitted by the module to a downstream module(see figure~\ref{fig:sourcembc}.)

\par

\begin{figure}[ht]
\centering
\epsfig{file=integration-module-output-channel.eps, height=.6in, width=1.75in}
\caption{Source Module Basic Components}
\label{fig:sourcembc}
\end{figure}

\par

\subsection{Processor}
A processor is the module that receives data from a source or a previous processor
module's output, performs the transformation operation and sends the data
into a sink module or a downstream processor module. The basic processor is
comprised of both ``input'' and an ``output'' connector channels and the data processing component.
The input channel receives all data from the upstream module and dispatches them to
the data processing component(see figure~\ref{fig:processormbc}.) It is the responsibility of
this component to transform the data. The transformed data is then sent to the downstream module
via the output channel.

\par

\begin{figure}
\centering
\epsfig{file=module-processor.eps, height=.6in, width=2.5in}
\caption{Processor Module Basic Components}
\label{fig:processormbc}
\end{figure}

\par

\subsection{Sink}
Sink modules convert and deliver data out of the stream in a format consumable by
an external application.  There are two types of sinks: analytic and delegate.
An analytic sink is used to perform analytic operations (such as count, gauge) on the
incoming data and store the result into a metric repository(See section~\ref{sec:Analytics}.)
A delegate sink translates data to the format expected by the external application.
After transforming the data, the resulting data is sent to the external application.

\par

The basic sink is comprised of a ``input'' channel connector and a data processing
component. The input channel receives data from the stream and dispatches
them to the data processing component which is responsible for connecting to the external
application(see figure~\ref{fig:sinkmbc}.) The sinks included in Spring XD have
configurable options for retries in case of failure.

\par

\begin{figure}
\centering
\epsfig{file=integration-module-input-channel.eps, height=.6in, width=1.75in}
\caption{Sink Module Basic Components}
\label{fig:sinkmbc}
\end{figure}

\par

\subsection{Job}
Spring XD uses Spring Batch \cite{spring-batch-reference}, a JSR standard (JSR-352)
for batch workload data processing as the foundation for implementing
job modules. A job enables users to execute enterprise batch processes within Spring XD.
Jobs are typically used when running long lasting tasks that have transactional requirements.
To account for failure scenarios, the workflow in the job can be designed to restart and 
resume operation or roll-back the transaction altogether. A job can be triggered by the
stream with the data that act as the input to start the batch processing. This makes
streams and job modules unified under a single platform.

\par

A job is typically comprised of a job definition along with the supporting
data processing components as shown in figure~\ref{fig:batchmbc}.
In some cases the job definition alone is sufficient to implement the desired behavior.

\par

\begin{figure}
\centering
\epsfig{file=integration-batch.eps, height=.8in, width=2in}
\caption{Batch Basic Components}
\label{fig:batchmbc}
\end{figure}

\par 

\subsection{Composite}
A composite module provides a way to create a single module comprised of
multiple modules (processing chain) together.  A composite module can be used to prevent
duplication when a processing chain of modules is used frequently.
Performance is another reason to use composite modules, because messages between the 
modules that comprise the composite module will be transmitted in memory vs. the message 
bus.  

\par

\subsection{Module Registry}
Module Registry is the place where Spring XD looks for the modules for
deployment. A module can be bundled as an archive or defined along with its
dependencies inside the module registry. The modules are defined in
the modules directory and are segregated in subdirectories by
type: \texttt{modules/source}, \texttt{modules/processor},
\texttt{modules/sink} and \texttt{modules/job}. The module registry is
configurable and one can upload the modules to module registry via REST request.
During the deployment of the module, Spring XD runtime will load the modules
dynamically from the module registry.


\section{Analytics}
\label{sec:Analytics}

Spring XD supports analytics functionality through various sink and processor modules.
Typically a ``main'' stream which obtains data from an external source (via
a source module) and writes it to an external data store (via a sink module)
is ``tapped'' -- meaning that a separate stream is created using the data
flowing from the first stream. The existence of this tap is unknown by the
main stream. This allows for analytics to be performed in a non-invasive manner.
The results of the analytics can be written to another sink module or a
subsequent processing module for further analysis.

\par

Spring XD includes analytics modules such as counter, field value counter,
aggregate counter, gauge and rich gauge. It also provides support for running
predictive analytics using PMML model scoring. Spring XD also includes a module
for executing arbitrary shell commands. This shell module can execute an R or
Python based processor or sink implementation which could perform dynamic model
scoring. Additionally, a Spark MLlib job may be executed as a Spring XD job,
benefiting from Spring XD's capability to monitor and manage the job.

\subsection {Basic Analytics}

To perform some of the basic analytic operations, Spring XD provides the following
modules. Since these modules write the analytic results to a data store, they are
considered sinks. Currently in-memory and Redis are supported data stores. This
data is exposed by the admin server via the REST API. This allows for simple
development of an application to gain access to the analytic data results.

\subsubsection {Counter}

This module counts the number of events triggered on various stages of a stream.
This can be used to tap a main stream at various stages and count the number
of messages sent out at each stage.

\subsubsection {Field Value Counter}

This module counts the number of occurrences of a specific field from a message
flowing through a stream. Spring XD supports the following message payload types
out of the box: POJO (Java Bean), Tuple and JSON string.

\subsubsection {Aggregate Counter}

The aggregate counter is similar to the simple counter but also keeps track of
the time period. Total count values for each minute, hour, day and month
of the period in which data was collected may be queried.

\subsubsection {Gauge}
Gauge is a metric that represents a single long value associated with a unique name.
In Spring XD, the gauge metric sink expects a numeric value as a payload.

\subsubsection {Rich Gauge}
Rich Gauge is a metric that holds a double value associated with a unique name. In
addition to the value, this metric keeps a running average along with the minimum and
maximum values and the count.

\subsection {Predictive Analytics using PMML}
Spring XD provides support for running real time model scoring using JPMML-Evaluator.
This evaluator supports a wide range of model types and is interoperable with
models exported from R, Rattle, KNIME, and RapidMiner. Once the analytical model
is defined as a PMML model, the evaluator can run model scoring based on the 
input data being ingested.

\par

The Shell based processor/sink implementation can be used to run predictive
analytics algorithms written in other languages such as Python and R which dynamically
update the model which makes predictive analytics more adaptive.

\subsection{Spark MLlib}
A Spark MLlib algorithm can be run as a sparkApp job in XD. This makes the
orchestration of job workflow, monitoring and management easier.
A Spark MLlib job can also be triggered by a stream with data that
would act as the input data for the MLlib algorithm. This closed loop
integration (integration streaming data to trigger the job) is very powerful
for the predictive analytics.


\section {Monitoring and Management}
 Spring XD makes it easier to manage and monitor the XD cluster and the
runtime components by providing useful metrics on those components and
user interface (shell and GUI) to interact with them.

\subsection {Monitoring via JMX and HTTP}
Spring XD supports configuring, monitoring and managing the runtime components
over HTTP and JMX. The admin, container, stream/job module components' runtime health,
environment, metrics can be accessed via the REST endpoints exposed by 
Spring Boot \cite{spring-boot}. One can also use JMX console to monitor and manage
the runtime components. The JMX metrics can also be obtained over HTTP using Jolokia.

\subsection {Management GUI}
Spring XD comes with an Admin GUI which lets the user manage the cluster environment,
stream and job monitoring and management which include monitoring the cluster health
with the deployed streams/jobs, workflows such as creation of batch jobs,
deployment of streams and jobs with runtime properties, monitoring the execution
of batch job, stop/restart the batch job when needed.

\subsection {Shell Interface}
Spring XD has the shell interface which is the main entry point for the user
to interact with XD admin to create/deploy and manage streams/jobs. This
interface implements all the REST endpoints exposed by Spring XD.



\input{spring-xd-deployment.tex}

\input{spring-xd-use-cases.tex}

\section{Related Work}
This section compares and contrasts Spring XD to similar projects.

\subsection{Spring XD and Spark}
Spark\cite{spark} is a general-purpose framework for large scale data processing.

For customers who need a reliable ingestion platform, analytical stream APIs, RESTful control over stream processing and batch jobs, as well as simplifying development and testing of Spark applications, Spring XD is available and supported today.

The following features differentiate Spring XD and Spark Streaming.

\begin{itemize*}
\item Ability to orchestrate relaible data pipelines that run 24/7.
\item Ingest data into HDFS complying with best practices (no small files), with no coding.
\item Ability to microbatch based on event count.
\item Data pieplines that process one event at a time.
\item Flexibility to specify hosts to dictate where data computations should be happening.
\item Decouple infrastructure code from business logic\slash analytics.
\end{itemize*}

The following features differentiate Spring XD and Spark Batch processing.

\begin{itemize*}
\item Provides a REST API and lifecycle management of Spark jobs.
\item Remains extensible to integrate with other Batch systems.
\end{itemize*}

Spring XD supports running Spark applications such as Streaming, MLLib, SparkSQL as job modules.
Spark users may implement the computation logic and leave the setup and launching of the
application to Spring XD.

Spring XD also supports running spark streaming application as an XD
module. The module acts as the driver while the computation defined by the streaming application
is executed in the spark cluster. The driver failure is automatically handled by Spring XD
by re-deploying the module to the eligible XD container.

\subsection{Spring XD and Storm}
Storm\cite{storm} is a distributed computation system for real time stream processing.

For customers who are in need of real time stream processing, distributed data computations, and ETL capabilities, Spring XD as an unified runtime supports variety of use cases ranging from classic enterprise to Big Data and IoT. 

The following features differentiate Spring XD and Storm.

\begin{itemize*}
\item Spring XD's Shell compared with Storm's API programming paradigm delivers immediate productivity. No coding, IDE, bundling or packaging necessary. The high level DSL abstracts complexities through developer friendly fixtures.
\item REST APIs allows topology to be built in isolation without having to disrupt existing pipelines.
\item Loosely coupled `modules' that are responsible for ingestion, analytics, data processing, machine learning or data export can be individually managed and dynamically scaled.
\item Simplified governance model for `modules' (unit\-of\-work) and colocation capabilities.
\item Composition of `modules' (unit\-of\-work) to improve performance characteristics. 
\item Building upon the functional stream processing model, users have the option to choose from Reactor\cite{reactor}, Spark Streaming or RxJava APIs, to build complex data centric applications.
\end{itemize*}

Spring XD provides you the right tool for the job, not forcing unnecessary complexity while remaining open and extensible runtime.

\subsection{Spring XD and Flume}
Flume\cite{flume} is a distributed system for collecting, aggregating and moving large data sets. 

For customers who need collecting and moving data, Spring XD simplifies lifecycle management, coordination, and automation of data pipelines. A high level DSL is all you need to operationalize data movements. 

The following features differentiate Spring XD and Flume.

\begin{itemize*}
\item The high level DSL allows you to build streams and jobs, with no coding involved. One line DSL declaration automates data collection, processing, and export to a desired format or writing to a specific data store.
\item Manage data centric workflows with REST APIs either via DSL, Admin UI, or custom dashboards.
\item Administer or monitor data pipelines through the UI, Jolokia or JMX endpoints. 
\item Thousands of combinations of data pipelines/flows can be created out of the box.
\item Granular controls to manifest batch job and step execution to create complex data driven workflows.
\item Flexibility through `Deployment Manifest' to declaratively configure data partitioning strategy to route data to a specific consumer instance in the cluster. Improves performance characteristics by the virtue of colocation.
\end{itemize*}

Spring XD's developer friendly fixtures deliver productivity. Flume offers HBase, Solr, and ElasticSearch sinks along with encryption support for Avro sources, which we are planning to address in our future releases.

\subsection{Spring XD and Oozie}
Oozie\cite{oozie} is a workflow scheduler engine to manage Hadoop \cite{hadoop} workloads such as MapReduce or Pig jobs. 

For customers who need workflow coordinator, Spring XD, as a unified platform provides orchestration of directed graph processes to govern not just Hadoop but any batch job workflows. 

The following features differentiate Spring XD and Oozie.

\begin{itemize*}
\item Building upon Spring Batch, a JSR standardization (JSR-352) of batch workload data processing, Spring XD inherits rich enterprise features to schedule abd execute batch job.
\item Out of the batch jobs such as file-to-jdbc, file-to-hdfs, ftp-to-hdfs, hdfs-to-jdbc, hdfs-to-mongo, jdbc-to-hdfs, spark-job, and sqoop-job.
\item Flexibility to create custom workflow jobs through REST APIs.
\item Dynamic scaling of out of the box workflow-jobs and custom-workflow-jobs without having to bring down the runtime or the currently running topology.
\item Runtime offers bidirectionality between real-time streaming and offline batch workflows to accommodate complex data processing use cases.
\item Ability to create and launch workflow-jobs from Admin UI. Historical snapshots of execution, errors and states are available for exploration via Admin UI.
\item Portable runtime that can run where there is JVM. On-prem, Pivotal Cloud Foundry, YARN, Mesos, Docker or EC2.
\end{itemize*}

Bridging the gaps between offline and near real time data, Spring XD will continue to evolve as one stop platform for data workflows to efficiently process data in-transit, at-rest, and in-use. Oozie offers HCatalog integration, which we are planning to address in our future releases.

\subsection{Spring XD and Sqoop}
Sqoop\cite{sqoop} assists with data transmission between Hadoop and relational databases.

For customers who need data ingest and export between various databases and Hadoop, Spring XD's unified platform facilitates out of the box batch jobs to orchestrate data movements. 

The following features differentiate Spring XD and Sqoop.

\begin{itemize*}
\item Comprehensive lifecycle management platform for not just data transmission between Hadoop and Databases but also real time data ingest, analytics and export use cases.
\item Extending the batch workflow infrastructure to write custom tasklets.
\item High level configuration DSL to create, deploy and destroy data pipelines.
\item Flexibility to introduce custom data pipelines through REST APIs.
\item Unified functional programming model support to build reactive-style data pipelines.
\end{itemize*}

Sqoop offers data validation, data merge, incremental data imports, and HCatalog integration among others. As a unified platform, Spring XD provides an out of the box Sqoop job to take advantage of the features at the same time also efficiently orchestrate data workload use cases. 


\input{spring-xd-conclusion.tex}

%\end{document}  % This is where a 'short' article might terminate

%ACKNOWLEDGMENTS are optional
%\section{Acknowledgments}
%This section is optional; it is a location for you
%to acknowledge grants, funding, editing assistance and
%what have you.  In the present case, for example, the
%authors would like to thank Gerald Murray of ACM for
%his help in codifying this \textit{Author's Guide}
%and the \textbf{.cls} and \textbf{.tex} files that it describes.

%
% The following two commands are all you need in the
% initial runs of your .tex file to
% produce the bibliography for the citations in your paper.
\bibliographystyle{abbrv}
\bibliography{spring-xd}  % sigproc.bib is the name of the Bibliography in this case
% You must have a proper ".bib" file
%  and remember to run:
% latex bibtex latex latex
% to resolve all references
%
% ACM needs 'a single self-contained file'!
%

% This next section command marks the start of
% Appendix B, and does not continue the present hierarchy
\balancecolumns
% That's all folks!
\end{document}
